\documentclass{article}
\usepackage[frenchb]{babel}
\usepackage{amsfonts}
\usepackage{amsmath}
\usepackage[T1]{fontenc}
\usepackage[utf8]{inputenc}
\usepackage{amsthm}
\usepackage{graphicx}
\usepackage{subfigure}
\usepackage{hyperref}



\title{Note de synthèse}
\author{Dell'Aiera Clément, Prévosteau Clément, David Wahiche}
\date{}

\newtheorem{definition}{Def}
\newtheorem{thm}{Théorème}
\newtheorem{ex}{Exercice}
\newtheorem{lem}{Lemme}
\newtheorem{dem}{Preuve}
\newtheorem{prop}{Proposition}
\newtheorem{cor}{Corollaire}

\newcommand{\Z}{\mathbb Z}
\newcommand{\R}{\mathbb R}
\newcommand{\C}{\mathbb C}
\newcommand{\Hil}{\mathcal H}
\newcommand{\Mn}{\mathcal M _n (\mathbb C)}

\begin{document}
\maketitle


Ce projet a pour objectif de présenter une introduction au réseau de neurones artificiels, et à certaines nouvelles méthodes qui sont récement appliquées en ce domaine, à savoir les méthodes de la géométrie de l'information, ainsi que l'apprentissage profond, ou \text{deep learning}.\\

\section{Déroulement du GT}

Notre projet a connu trois périodes. Les 3 premiers mois ont été des mois de prise en main du sujet, notamment d'apprentissage des techniques de réseau de neurones classiques, perceptron par exemple. Il a bien sûr fallu apprendre un peu de géométrie différentielle, et lire les articles proposés par l'encadrant. \\

Après décembre, le début du rapport a été rédigé et les premières expérimentation avec le code ont été faîtes. Nous avons alors codé une classe python entièrement à la main pour manier les réseaux de neurones à pluiseurs couches. C'est à ce moment que nous avons comparé ces méthodes avec celles plus classiques vu en cours à l'ENSAE, comme les SVM par exemple.\\

 Au final, la dernière partie de l'année a été consacrée à l'application des méthodes de géométrie de l'information, ainsi qu'aux techniques d'apprentissage profond. Cette période a été la plus motivante, toutes les différentes notions ont convergé, et nous navons pu regretter qu'une seule chose : que l'année ne soit plus longue pour pouvoir appronfondir le sujet. En effet, notre but ultime aurait été de coder une descente de gradient riemannien sur des réseaux de neurones pré-entrâinés avec des \textit{Deep Belief Networks}, et d'incorporer le côté riemannien aux machines de Boltzman. Du point de vue des difficultés, nous en avons surtout rencontrées au niveau de la littérature. Le domaine est en effet très récent (certains articles datent de $2013$), et il est difficile de trouver des exposés clairs et concis, ni d'harmonisation des concepts ou notations. \\

\section{Synthèse des travaux}

\subsection{Réseaux de neurones}
Les réseaux de neurones peuvent être rapidement présentés commedes modèles de régression. Ils consistent en plusieurs couches de neurones empilées les une au-dessus des autres, reliées par des connexions dont la réponse dépend d'un poids et d'une fonction d'activation. Ces derniers sont les paramètres du modèle dont l'expérimentateur va faire varier la valeur en entraînant le réseau sur une base de données via un apprentissage supervisé. \\

Formellement, un réseau de neurones est la donnée d'un entier $N$, le nombre de couches, de $N-1$ matrices de poids $W_j, \ j=1,N-1$ et d'autant de fonctions d'activations $\phi_j$. Si $x\in D$. Alors, si l'entrée est une donnée $x\in \mathcal D$, le réseau agit récursivement :
\[\left\{\begin{array}{c}x_0=x\\ x_{j+1}=f_j(W_j x_j)\end{array}\right.\]
Dans la formule ci-dessus, la fonction d'activation est appliquée à chaque terme du vecteur $W_j x_j$. Les fonctions d'activation sont à choisir parmi celles connues des praticiens : \textit{tanh}, sigmoïde,... Les matrices de poids constituent les paramètres du modèle.\\

Ces réseaux s'utilisent en apprentissage supervisé, c'est-à-dire que l'on lui fournit un entrée $x$, et la réponse attendue $t$. Le réseau compare alors l'étiquette $t$ à la réponse $y$, et altère alors ses paramètres (les poids des connexions) selon un algorithme prédéfini, à savoir une descente de gradient sur une fonction de perte.\\

\subsection{Méthodes de géométrie de l'information}
C'est ici qu'intervient la géométrie de l'information. L'idée est de voir l'espace des paramètres comme une variété riemannienne, c'est-à-dire un espace muni d'un système de coordonnées et d'une métrique qui permet de donner une <<longueur>> à une variation des paramètres (intuitivement l'espace tangent est une variation infinitésimale de paramètres). On peut alors choisir des métriques <<invariantes>> ou <<intrinsèques>> ~\cite{Ollivier} au sens où elles dépendent que de ce que fait le réseau et non de son paramétrage. Par exemple, Ollivier décide d'utiliser la métrique qui est donnée par la matrice de Fisher : elle est symmétrique définie positive, c'est donc bien en chaque paramètre du modèle une métrique. Ollivier montre qu'elle est invariante. Nous avons donc testé des descentes de gradients riemanniens et observé une convergence plus rapide.\\

La géométrie de l'information utilise des outils plus compliqués. Dans le livre d'Amari sur le sujet ~\cite{Amari} figure une introduction aux connexions affines. Toutefois, bien qu'ayant commencé à les étudier, nous avons décidé de nous limiter au cadre de la descente de gradient adaptée, et ce pour plusieurs raisons. La première est que le niveau technique demandé, en géométrie notamment, est élevé, plus que ce que nous nous sentions capables de faire. Ensuite, il apparaît dans la littérature que ces méthodes plus complexes n'aboutissent pas des résultats statistiques probants, en tout cas pour le moment. \\

\subsection{Apprentissage profond ou \textit{deep learning}}

Les méthodes de \textit{deep learning} incorporent, en plus de réseaux de neurones à plusieurs couches, des techniques de préparations des poids des réseaux. Au moyen de se que l'on appelle des machines de Boltzman restreintes, qui sont des modèles issus de la physique statistique voisins par exemple du modèle d'Ising, on peut entraîner chaque couche avant d'effectuer la descente de gradient.\\

Ces technisques sont très actuelles, et bénéficient d'un intérêt croissant de la part des entreprises qui traitent beaucoup de données. Google par exemple, qui vient de racheter une société spécialisée dans ces techniques, Deepmind.\\

\section{Résultats}

Les algorithmes ont été testé sur le \textit{benchmark} de l'apprentissage : la base d'images de chiffres écrits à la main MNIST, disponible sur le site de Hinton, qui contient $70 000$ images, toutes au même format de $28\times 28$ pixels. Nous avons testés les méthodes imbriquées les unes dans les autres de manière de plus en plus complexes : au départ, un simple classifieur logistique sur MNIST, puis un réseau de neurones à $2$ couches dont la dernière est un classifieur logistique, enfin une préparation des données avec des machines de Boltzman, puis la même technique. Nous avons réussi à atteindre des taux d'erreurs sur l'échantillon de test de MNIST de, respectivement, $7,489\%$, $1,65\%$ et $1,34\%$. Pour comparer, le lecteur intéressé peut aller sur la page ~\url{http://yann.lecun.com/exdb/mnist/}, où il trouvera le score d'équipes de recherches. \\

Nous avons aussi testé les différences entre descente de gradient riemannien et descente de gradient classique, sur des exemples simulés tels que des ensembles aléatoires non linéairement séparables ( XOR dans le rapport). Nous aurions voulu adapter et faire le lien entre ces techniques et le cadre riemannien mais n'avons pas aboutit. \\



\bibliographystyle{plain}
\bibliography{biblio} 
\nocite{*}

\end{document} 




































